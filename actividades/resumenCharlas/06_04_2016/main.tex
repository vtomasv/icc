\documentclass[12pt,letterpaper]{article}
\usepackage[utf8]{inputenc}
\usepackage[spanish, es-tabla]{babel}
\usepackage[version=3]{mhchem}
\usepackage[journal=jacs]{chemstyle}
\usepackage{amsmath}
\usepackage{amsfonts}
\usepackage{amssymb}
\usepackage{makeidx}
\usepackage{xcolor}
\usepackage[stable]{footmisc}
\usepackage[section]{placeins}
%Paquetes necesarios para tablas
\usepackage{longtable}
\usepackage{array}
\usepackage{xtab}
\usepackage{multirow}
\usepackage{colortab}
%Paquete para el manejo de las unidades
\usepackage{siunitx}
\sisetup{mode=text, output-decimal-marker = {,}, per-mode = symbol, qualifier-mode = phrase, qualifier-phrase = { de }, list-units = brackets, range-units = brackets, range-phrase = --}
\DeclareSIUnit[number-unit-product = \;] \atmosphere{atm}
\DeclareSIUnit[number-unit-product = \;] \pound{lb}
\DeclareSIUnit[number-unit-product = \;] \inch{"}
\DeclareSIUnit[number-unit-product = \;] \foot{ft}
\DeclareSIUnit[number-unit-product = \;] \yard{yd}
\DeclareSIUnit[number-unit-product = \;] \mile{mi}
\DeclareSIUnit[number-unit-product = \;] \pint{pt}
\DeclareSIUnit[number-unit-product = \;] \quart{qt}
\DeclareSIUnit[number-unit-product = \;] \flounce{fl-oz}
\DeclareSIUnit[number-unit-product = \;] \ounce{oz}
\DeclareSIUnit[number-unit-product = \;] \degreeFahrenheit{\SIUnitSymbolDegree F}
\DeclareSIUnit[number-unit-product = \;] \degreeRankine{\SIUnitSymbolDegree R}
\DeclareSIUnit[number-unit-product = \;] \usgallon{galón}
\DeclareSIUnit[number-unit-product = \;] \uma{uma}
\DeclareSIUnit[number-unit-product = \;] \ppm{ppm}
\DeclareSIUnit[number-unit-product = \;] \eqg{eq-g}
\DeclareSIUnit[number-unit-product = \;] \normal{\eqg\per\liter\of{solución}}
\DeclareSIUnit[number-unit-product = \;] \molal{\mole\per\kilo\gram\of{solvente}}
\usepackage{cancel}
%Paquetes necesarios para imágenes, pies de página, etc.
\usepackage{graphicx}
\usepackage{lmodern}
\usepackage{fancyhdr}
\usepackage[left=4cm,right=2cm,top=3cm,bottom=3cm]{geometry}

%Instrucción para evitar la indentación
%\setlength\parindent{0pt}
%Paquete para incluir la bibliografía
\usepackage[backend=bibtex,style=chem-acs,biblabel=dot]{biblatex}
\addbibresource{references.bib}

%Formato del título de las secciones

\usepackage{titlesec}
\usepackage{enumitem}
\titleformat*{\section}{\bfseries\large}
\titleformat*{\subsection}{\bfseries\normalsize}

%Creación del ambiente anexos
\usepackage{float}
\floatstyle{plaintop}
\newfloat{anexo}{thp}{anx}
\floatname{anexo}{Anexo}
\restylefloat{anexo}
\restylefloat{figure}

%Modificación del formato de los captions
\usepackage[margin=10pt,labelfont=bf]{caption}

%Paquete para incluir comentarios
\usepackage{todonotes}

%Paquete para incluir hipervínculos
\usepackage[colorlinks=true,
            linkcolor = blue,
            urlcolor  = blue,
            citecolor = black,
            anchorcolor = blue]{hyperref}

%%%%%%%%%%%%%%%%%%%%%%
%Inicio del documento%
%%%%%%%%%%%%%%%%%%%%%%

\begin{document}
\renewcommand{\labelitemi}{$\checkmark$}

\renewcommand{\CancelColor}{\color{red}}

\newcolumntype{L}[1]{>{\raggedright\let\newline\\\arraybackslash}m{#1}}

\newcolumntype{C}[1]{>{\centering\let\newline\\\arraybackslash}m{#1}}

\newcolumntype{R}[1]{>{\raggedleft\let\newline\\\arraybackslash}m{#1}}

\begin{center}
  %%%%%%%%%%%%%%%%%%%%%%%%%%%%%%%%%%%%%%
  % Ingresa el mejor titulo que se te ocurra
  %%%%%%%%%%%%%%%%%%%%%%%%%%%%%%%%%%%%%%
	\textbf{\LARGE{La investigacion de Workfloes Jose A. Pino y Aidan Hogan en la web 3.0. }}\\
	\vspace{4mm}
		\textbf{\large{Tomás Vera\\
  email: \href{mailto:vtomasv@gmail.com}{vtomasv@gmail.com}  } }\\
	\vspace{3mm}
	\textbf{\large{Universidad de Chile}}\\
	\textbf{\small{CC71T-1 Investigación en Cs. de la Computación.(Métodos,Técnicas,Persp.) }}\\
	\textbf{\large{Profesor: Claudio Gutierrez \\
  email: \href{mailto:cgutierr@dcc.uchile.cl}{cgutierr@dcc.uchile.cl}  } }\\
	\today
\end{center}

%%%%%%%%%%%%%%%%%%%%%%%%%%%%%%%%%%%%%%
% Resumen formal de la noticia
%%%%%%%%%%%%%%%%%%%%%%%%%%%%%%%%%%%%%%
\section*{\centering Resumen}
El día 6 de Abril del 2016, en el contexto del curso de Introducción a la investigación de las ciencias de la computacion liderado por el profesor Claudio Gutierrez se realizo la presentación de los Doctores Jose A. Pino y el Doctor  Aidan Hogan de la Universidad de Chile.
La presentación introducía a los alumnos en las áreas generales de investigación del Doctor Jose A. Pino\autocite{JP} en Sistemas Colaborativos, Interacción Humano-Computador, Informática Educativa, estas investigaciones se llevan a cabo dentro del laboratorio Collaborative Applications Research Laboratory. Adicionalmente el Doctor Aidan Hogan\autocite{AG} comento sobre el futuro de la Web y parte sus investigaciones en Web Semántica y el impacto para la compresión de la web por las computadoras, sus trabajos de investigación están bajo el alero del laboratorio Center for Semantic Web Research\autocite{CSWR}.

%%%%%%%%%%%%%%%%%%%%%%%%%%%%%%%%%%%%%%
% Refinamiento de la idea
%%%%%%%%%%%%%%%%%%%%%%%%%%%%%%%%%%%%%%
\section{Jose A. Pino, llevando los flujos de trabajo al siguiente nivel}

Poder levantar, analizar, modelar y probar de forma cientifica Flujos de Trabajo (Workflows) no es tarea sencilla. El Doctor Jose A. Pino explica la complejidad y las distintas soluciones existentes por ejemplo el uso de la ingenieria inversa de logs (vitícolas de sistemas) para la re-construcción de workflows, reconociendo de patrones de workflows no comunes (flexibilidad), workflows orientados a patrones, levantamiento de workflows a travez de HistoryTelling y sus actuales campos de trabajo sobre la comparación de distintos modelos de modelado de workflows en otras disciplinas, construyendo nuevos modelos a partir de la utilización de un subconjunto de estos para cubrir las brechas no cubiertas por su individualidad.

\section{Aidan Hogan y el internet de las maquinas}

Actualmente para las personas leer informacion de internet y procesarla es de alguna manera natural, ya que estas estructuras de datos, imágenes y vínculos están pensadas para que las personas las usen. Sin embargo las computadoras no tiene la habilidad de entender estos datos de forma natural, la web semántica busca poder hacer que el internet no solo sea entendible por los humanos sino también por las maquinas y ahora esta informacion tenga sentido también para los procesos automáticos que los computadores pueden realizar y de esta manera puedan generar nuevo contenido, por ejemplo entender informacion relevante a un producto en particular y poder compara el precio de los mismos sin la necesidad de que ese comportamiento haya sido desarrollado por un humano con anterioridad.

%%%%%%%%%%%%%%%%%%%%%%%%%%%%%%%%%%%%%%
% Referencias sobre la noticia
%%%%%%%%%%%%%%%%%%%%%%%%%%%%%%%%%%%%%%
\section{Referencias\label{sec:references}}

\printbibliography[heading=none]

\end{document}
