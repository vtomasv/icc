\documentclass[12pt,letterpaper]{article}
\usepackage[utf8]{inputenc}
\usepackage[spanish, es-tabla]{babel}
\usepackage[version=3]{mhchem}
\usepackage[journal=jacs]{chemstyle}
\usepackage{amsmath}
\usepackage{amsfonts}
\usepackage{amssymb}
\usepackage{makeidx}
\usepackage{xcolor}
\usepackage[stable]{footmisc}
\usepackage[section]{placeins}
%Paquetes necesarios para tablas
\usepackage{longtable}
\usepackage{array}
\usepackage{xtab}
\usepackage{multirow}
\usepackage{colortab}
%Paquete para el manejo de las unidades
\usepackage{siunitx}
\sisetup{mode=text, output-decimal-marker = {,}, per-mode = symbol, qualifier-mode = phrase, qualifier-phrase = { de }, list-units = brackets, range-units = brackets, range-phrase = --}
\DeclareSIUnit[number-unit-product = \;] \atmosphere{atm}
\DeclareSIUnit[number-unit-product = \;] \pound{lb}
\DeclareSIUnit[number-unit-product = \;] \inch{"}
\DeclareSIUnit[number-unit-product = \;] \foot{ft}
\DeclareSIUnit[number-unit-product = \;] \yard{yd}
\DeclareSIUnit[number-unit-product = \;] \mile{mi}
\DeclareSIUnit[number-unit-product = \;] \pint{pt}
\DeclareSIUnit[number-unit-product = \;] \quart{qt}
\DeclareSIUnit[number-unit-product = \;] \flounce{fl-oz}
\DeclareSIUnit[number-unit-product = \;] \ounce{oz}
\DeclareSIUnit[number-unit-product = \;] \degreeFahrenheit{\SIUnitSymbolDegree F}
\DeclareSIUnit[number-unit-product = \;] \degreeRankine{\SIUnitSymbolDegree R}
\DeclareSIUnit[number-unit-product = \;] \usgallon{galón}
\DeclareSIUnit[number-unit-product = \;] \uma{uma}
\DeclareSIUnit[number-unit-product = \;] \ppm{ppm}
\DeclareSIUnit[number-unit-product = \;] \eqg{eq-g}
\DeclareSIUnit[number-unit-product = \;] \normal{\eqg\per\liter\of{solución}}
\DeclareSIUnit[number-unit-product = \;] \molal{\mole\per\kilo\gram\of{solvente}}
\usepackage{cancel}
%Paquetes necesarios para imágenes, pies de página, etc.
\usepackage{graphicx}
\usepackage{lmodern}
\usepackage{fancyhdr}
\usepackage[left=4cm,right=2cm,top=3cm,bottom=3cm]{geometry}

%Instrucción para evitar la indentación
%\setlength\parindent{0pt}
%Paquete para incluir la bibliografía
\usepackage[backend=bibtex,style=chem-acs,biblabel=dot]{biblatex}
\addbibresource{references.bib}

%Formato del título de las secciones

\usepackage{titlesec}
\usepackage{enumitem}
\titleformat*{\section}{\bfseries\large}
\titleformat*{\subsection}{\bfseries\normalsize}

%Creación del ambiente anexos
\usepackage{float}
\floatstyle{plaintop}
\newfloat{anexo}{thp}{anx}
\floatname{anexo}{Anexo}
\restylefloat{anexo}
\restylefloat{figure}

%Modificación del formato de los captions
\usepackage[margin=10pt,labelfont=bf]{caption}

%Paquete para incluir comentarios
\usepackage{todonotes}

%Paquete para incluir hipervínculos
\usepackage[colorlinks=true,
            linkcolor = blue,
            urlcolor  = blue,
            citecolor = black,
            anchorcolor = blue]{hyperref}

%%%%%%%%%%%%%%%%%%%%%%
%Inicio del documento%
%%%%%%%%%%%%%%%%%%%%%%

\begin{document}
\renewcommand{\labelitemi}{$\checkmark$}

\renewcommand{\CancelColor}{\color{red}}

\newcolumntype{L}[1]{>{\raggedright\let\newline\\\arraybackslash}m{#1}}

\newcolumntype{C}[1]{>{\centering\let\newline\\\arraybackslash}m{#1}}

\newcolumntype{R}[1]{>{\raggedleft\let\newline\\\arraybackslash}m{#1}}

\begin{center}

  %%%%%%%%%%%%%%%%%%%%%%%%%%%%%%%%%%%%%%
  % Ingresa el mejor titulo que se te ocurra
  %%%%%%%%%%%%%%%%%%%%%%%%%%%%%%%%%%%%%%
	\textbf{\LARGE{ 5 posibles formas de romper la seguridad de tu iPhone }}\\
	\vspace{3mm}
		\textbf{{Tomás Vera\\
  \small{email: \href{mailto:vtomasv@gmail.com}{vtomasv@gmail.com}  } }}\\
	\vspace{3mm}
	\textbf{\large{Universidad de Chile}}\\
	\textbf{\small{CC71T-1 Investigación en Cs. de la Computación.(Métodos,Técnicas,Persp.) }}\\
	\textbf{\large{Profesor: Claudio Gutierrez \\
  \small{email:\href{mailto:cgutierr@dcc.uchile.cl}{cgutierr@dcc.uchile.cl}  } }}\\
	\today
\end{center}

%%%%%%%%%%%%%%%%%%%%%%%%%%%%%%%%%%%%%%
% Resumen formal de la noticia
%%%%%%%%%%%%%%%%%%%%%%%%%%%%%%%%%%%%%%
\section*{\centering Resumen}
El FBI desistió de una demanda que había interpuesto a Appel Inc en la cual solicitaba ayuda para poder romper la seguridad del iPhone 5c del terrorista  Syed Farook. El FBI aun no ha declarado cual fue la forma en la cual pudieron romper la seguridad del iPhone\autocite{web:1}, sin embargo, 9 expertos de seguridad móvil y especialistas forenses coinciden con 5 alternativas posibles para poder lograr este objetivo. Es posible enumerar las alternativas como, utilización de una vulnerabilidad de día cero, evadir la solicitud de clave del sistema operativo, reiniciar la memoria RAM en forma recurrente para realizar fuerza bruta, interviniendo fisicamente los circuitos del iPhone para saltar su seguridad y escuchar como su fuera una caja fuerte al iPhone a fin descubrir la complejidad de la contraseña utilizada; las alternativas están enumeradas de menor esfuerzo/complejidad a mayor esfuerzo/complejidad.




%%%%%%%%%%%%%%%%%%%%%%%%%%%%%%%%%%%%%%
% Tu voz sobre el impacto de esta noticia
%%%%%%%%%%%%%%%%%%%%%%%%%%%%%%%%%%%%%%
\section{Opinion}
Si bien la noticia parece tener un sentido netamente tecnológico, su fondo, esta relacionado a la privacidad de las personas en el mundo digital. Si es posible romper la seguridad de un teléfono personal, donde queda la privacidad de las personas. Dado esto es que las distintas organizaciones impulsoras por la defensa de la privacidad han reforzado sus esfuerzos por comunicar y educar como escribir password seguras y que tipo de software se debe utilizar para no perder este preciado derecho.
Al terminar este informe se declaro que el FBI utilizo un exploit de cero día construido por Gray-Hat\autocite{web:2}, se estima que el costo del mismo supero el millón de dólares. 

%%%%%%%%%%%%%%%%%%%%%%%%%%%%%%%%%%%%%%
% Referencias sobre la noticia
%%%%%%%%%%%%%%%%%%%%%%%%%%%%%%%%%%%%%%
\section{Referencias\label{sec:references}}

\printbibliography[heading=none]



\end{document}
