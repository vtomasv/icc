\documentclass[12pt,letterpaper]{article}
\usepackage[utf8]{inputenc}
\usepackage[spanish, es-tabla]{babel}
\usepackage[version=3]{mhchem}
\usepackage[journal=jacs]{chemstyle}
\usepackage{amsmath}
\usepackage{amsfonts}
\usepackage{amssymb}
\usepackage{makeidx}
\usepackage{xcolor}
\usepackage[stable]{footmisc}
\usepackage[section]{placeins}
%Paquetes necesarios para tablas
\usepackage{longtable}
\usepackage{array}
\usepackage{xtab}
\usepackage{multirow}
\usepackage{colortab}
%Paquete para el manejo de las unidades
\usepackage{siunitx}
\sisetup{mode=text, output-decimal-marker = {,}, per-mode = symbol, qualifier-mode = phrase, qualifier-phrase = { de }, list-units = brackets, range-units = brackets, range-phrase = --}
\DeclareSIUnit[number-unit-product = \;] \atmosphere{atm}
\DeclareSIUnit[number-unit-product = \;] \pound{lb}
\DeclareSIUnit[number-unit-product = \;] \inch{"}
\DeclareSIUnit[number-unit-product = \;] \foot{ft}
\DeclareSIUnit[number-unit-product = \;] \yard{yd}
\DeclareSIUnit[number-unit-product = \;] \mile{mi}
\DeclareSIUnit[number-unit-product = \;] \pint{pt}
\DeclareSIUnit[number-unit-product = \;] \quart{qt}
\DeclareSIUnit[number-unit-product = \;] \flounce{fl-oz}
\DeclareSIUnit[number-unit-product = \;] \ounce{oz}
\DeclareSIUnit[number-unit-product = \;] \degreeFahrenheit{\SIUnitSymbolDegree F}
\DeclareSIUnit[number-unit-product = \;] \degreeRankine{\SIUnitSymbolDegree R}
\DeclareSIUnit[number-unit-product = \;] \usgallon{galón}
\DeclareSIUnit[number-unit-product = \;] \uma{uma}
\DeclareSIUnit[number-unit-product = \;] \ppm{ppm}
\DeclareSIUnit[number-unit-product = \;] \eqg{eq-g}
\DeclareSIUnit[number-unit-product = \;] \normal{\eqg\per\liter\of{solución}}
\DeclareSIUnit[number-unit-product = \;] \molal{\mole\per\kilo\gram\of{solvente}}
\usepackage{cancel}
%Paquetes necesarios para imágenes, pies de página, etc.
\usepackage{graphicx}
\usepackage{lmodern}
\usepackage{fancyhdr}
\usepackage[left=4cm,right=2cm,top=3cm,bottom=3cm]{geometry}

%Instrucción para evitar la indentación
%\setlength\parindent{0pt}
%Paquete para incluir la bibliografía
\usepackage[backend=bibtex,style=chem-acs,biblabel=dot]{biblatex}
\addbibresource{references.bib}

%Formato del título de las secciones

\usepackage{titlesec}
\usepackage{enumitem}
\titleformat*{\section}{\bfseries\large}
\titleformat*{\subsection}{\bfseries\normalsize}

%Creación del ambiente anexos
\usepackage{float}
\floatstyle{plaintop}
\newfloat{anexo}{thp}{anx}
\floatname{anexo}{Anexo}
\restylefloat{anexo}
\restylefloat{figure}

%Modificación del formato de los captions
\usepackage[margin=10pt,labelfont=bf]{caption}

%Paquete para incluir comentarios
\usepackage{todonotes}

%Paquete para incluir hipervínculos
\usepackage[colorlinks=true,
            linkcolor = blue,
            urlcolor  = blue,
            citecolor = black,
            anchorcolor = blue]{hyperref}

%%%%%%%%%%%%%%%%%%%%%%
%Inicio del documento%
%%%%%%%%%%%%%%%%%%%%%%

\begin{document}
\renewcommand{\labelitemi}{$\checkmark$}

\renewcommand{\CancelColor}{\color{red}}

\newcolumntype{L}[1]{>{\raggedright\let\newline\\\arraybackslash}m{#1}}

\newcolumntype{C}[1]{>{\centering\let\newline\\\arraybackslash}m{#1}}

\newcolumntype{R}[1]{>{\raggedleft\let\newline\\\arraybackslash}m{#1}}

\begin{center}
  %%%%%%%%%%%%%%%%%%%%%%%%%%%%%%%%%%%%%%
  % Ingresa el mejor titulo que se te ocurra
  %%%%%%%%%%%%%%%%%%%%%%%%%%%%%%%%%%%%%%
	\textbf{\LARGE{Listados de preguntas/problemas}}\\
	\vspace{4mm}
		\textbf{\large{Tomás Vera\\
  email: \href{mailto:vtomasv@gmail.com}{vtomasv@gmail.com}  } }\\
	\vspace{3mm}
	\textbf{\large{Universidad de Chile}}\\
	\textbf{\small{CC71T-1 Investigación en Cs. de la Computación.(Métodos,Técnicas,Persp.) }}\\
	\textbf{\large{Profesor: Claudio Gutierrez \\
  email: \href{mailto:cgutierr@dcc.uchile.cl}{cgutierr@dcc.uchile.cl}  } }\\
	\today
\end{center}

%%%%%%%%%%%%%%%%%%%%%%%%%%%%%%%%%%%%%%
% Resumen formal de la noticia
%%%%%%%%%%%%%%%%%%%%%%%%%%%%%%%%%%%%%%
\section*{\centering Resumen}
Una vez evaluadas las tres preguntas/problemas presentados en el informe anterior[1] se debe seleccionar una para ser refinada. La actividad de refinamiento o detalle de una pregunta/problema se basa en una pauta entregada por el profesor Claudio Gutierrez en clase.

%%%%%%%%%%%%%%%%%%%%%%%%%%%%%%%%%%%%%%
% Refinamiento de la idea
%%%%%%%%%%%%%%%%%%%%%%%%%%%%%%%%%%%%%%
\section{Recuperar autos robados en Chile}

Uno de los delitos más comunes en Latinoamérica son los robos de vehículos\autocite{robos}, en Santiago la tasa de recuperación de vehículos es cercana al 80\% \autocite{recupero}, sin embargo antes de ser recuperados son utilizados para realizar otros tipos de ilícitos lo que perjudica a sus dueños reales. Poder contar con un mecanismo de recuperación temprana permitirá bajar la tasa de delitos efectuados con autos robados. Una de las tecnicas utilizada por las policías de Santiago de Chile es hacer uso de cámaras inteligentes\autocite{recono} que verifican en tiempo real si la placa patente del vehículo que los precede esta reportada como robada o no, contar con este mecanismo de manera universal permitirá contar con mas puntos de observación que faciliten las tareas de identificación y recuperación de vehículos robados.

\subsection{Titulo}
Reducir los tiempos de recuperación de los automóviles sustraídos en la Region Metropolitana, mediante un modelo colaborativo de detección de vehículos sustraídos.

\subsection{Contexto}
En los últimos 15 años el parque automotor se ha septuplicado. Desde 2010 hasta el año pasado, el incremento alcanzó un 41\%, superando los 7 millones de vehículos inscritos a nivel nacional\autocite{crecimiento}.
La mayor concentración de éstos se encuentra en la Región Metropolitana, que desde 1990 registra 4.121.355 automóviles.
De acuerdo a un estudio presentado por Prose-Chile\autocite{robos_2015} el martes 19 de enero del 2016, durante el año 2015, el robo de autos asegurados aumentó un 4,8\% en relación con el año anterior. Asimismo, en cuanto al hurto de vehículos de alta gama, cuyo valor sobrepasa los 20 millones de pesos, se registró un crecimiento de 72,3\%, casi cuatro puntos mas que el año 2014.
Según datos del Ministerio, el 75\% de los autos robados son finalmente encontrados por las policías.

\subsection{Problema}
Altos tiempos de denuncio y recuperación de vehículos sustraídos en la region metropolitana.


\subsection{Aspectos relevantes y efectos de robos de vehículos en la region metropolitana}
Si bien la tasa de recupero de vehículos robados en la region metropolitana es alta, estos vehículos antes de ser recuperados son utilizados para realizar otro tipo de ilícitos los cuales perjudican a sus dueños reales con eventos como, infracciones, daños a terceros o a propio vehículo sustraído u otros problemas como tramites engorrosos para verificar que no fueron los autores de los ilícitos cometidos con su automóvil.

\subsection{Como medir la calidad de las soluciones a este problema}
Cada una de las soluciones planteadas deben permitir disminuir los tiempos de recupero de los vehículos sustraídos, para eso se establecen las siguientes reglas.

\begin{enumerate}
\item Existe un tiempo normal de recupero de vehículos, se busca reducir en al menos este tiempo en un 40\%
\item El tiempo promedio para realizar un denuncio de un vehículo robado hoy tiene un tiempo T el cual se espera una mejora del 30\%
\item Se debe bajar la tasa de robos de autos para la comisión de delitos en un 10\%
\end{enumerate}

\subsection{Tipo de problema}
Poder contar con un mecanismo (humano o artificial) lo suficientemente amplio como para poder detectar tempranamente a un vehículo robado.

\subsection{Soluciones hoy en día}
En la actualidad existe una solucion que permiten realizar una denuncia de forma rápida. La aplicación se puede descargar en celulares y computadores y tiene como objetivo reducir los tiempos de reacción de las policías ante el robo de autos. También está disponible en el sitio web www.alertaauto.cl, desde donde se podrán registrar los vehículos y realizar las denuncias.
Otra solución es las utilizada por las policías de Santiago de Chile es hacer uso de cámaras inteligentes\autocite{recono} que verifican en tiempo real si la placa patente del vehículo que los precede esta reportada como robada o no.

\subsection{Falencias de soluciones actuales}
Con respecto a la solución de las policías siendo esta la mas efectiva su cobertura es escasa lo que no le permite una mayor efectividad, el mecanismos a su vez por su costo no es posible tenerlo en todas las patrullas existentes en la actualidad.
La segunda solución no permite detectar tempranamente autos robados, solo realizar el denuncio del mismo.


\subsection{Posible solución}
Utilización de la colaboración de otros conductores para la recuperación de los vehículos alertados como sustraídos por medio de una aplicacion social similar a Waze.



%%%%%%%%%%%%%%%%%%%%%%%%%%%%%%%%%%%%%%
% Referencias sobre la noticia
%%%%%%%%%%%%%%%%%%%%%%%%%%%%%%%%%%%%%%
\section{Referencias\label{sec:references}}

\printbibliography[heading=none]

\section{Tu opinion es muy importante!}
\begin{figure}
    \centering
    \includegraphics[width=4cm]{./images/vote.png}
    \captionsetup{justification=centering, singlelinecheck=false}
\end{figure}

\end{document}
