\documentclass[12pt,letterpaper]{article}
\usepackage[utf8]{inputenc}
\usepackage[spanish, es-tabla]{babel}
\usepackage[version=3]{mhchem}
\usepackage[journal=jacs]{chemstyle}
\usepackage{amsmath}
\usepackage{amsfonts}
\usepackage{amssymb}
\usepackage{makeidx}
\usepackage{xcolor}
\usepackage[stable]{footmisc}
\usepackage[section]{placeins}
%Paquetes necesarios para tablas
\usepackage{longtable}
\usepackage{array}
\usepackage{xtab}
\usepackage{multirow}
\usepackage{colortab}
%Paquete para el manejo de las unidades
\usepackage{siunitx}
\sisetup{mode=text, output-decimal-marker = {,}, per-mode = symbol, qualifier-mode = phrase, qualifier-phrase = { de }, list-units = brackets, range-units = brackets, range-phrase = --}
\DeclareSIUnit[number-unit-product = \;] \atmosphere{atm}
\DeclareSIUnit[number-unit-product = \;] \pound{lb}
\DeclareSIUnit[number-unit-product = \;] \inch{"}
\DeclareSIUnit[number-unit-product = \;] \foot{ft}
\DeclareSIUnit[number-unit-product = \;] \yard{yd}
\DeclareSIUnit[number-unit-product = \;] \mile{mi}
\DeclareSIUnit[number-unit-product = \;] \pint{pt}
\DeclareSIUnit[number-unit-product = \;] \quart{qt}
\DeclareSIUnit[number-unit-product = \;] \flounce{fl-oz}
\DeclareSIUnit[number-unit-product = \;] \ounce{oz}
\DeclareSIUnit[number-unit-product = \;] \degreeFahrenheit{\SIUnitSymbolDegree F}
\DeclareSIUnit[number-unit-product = \;] \degreeRankine{\SIUnitSymbolDegree R}
\DeclareSIUnit[number-unit-product = \;] \usgallon{galón}
\DeclareSIUnit[number-unit-product = \;] \uma{uma}
\DeclareSIUnit[number-unit-product = \;] \ppm{ppm}
\DeclareSIUnit[number-unit-product = \;] \eqg{eq-g}
\DeclareSIUnit[number-unit-product = \;] \normal{\eqg\per\liter\of{solución}}
\DeclareSIUnit[number-unit-product = \;] \molal{\mole\per\kilo\gram\of{solvente}}
\usepackage{cancel}
%Paquetes necesarios para imágenes, pies de página, etc.
\usepackage{graphicx}
\usepackage{lmodern}
\usepackage{fancyhdr}
\usepackage[left=4cm,right=2cm,top=3cm,bottom=3cm]{geometry}

%Instrucción para evitar la indentación
%\setlength\parindent{0pt}
%Paquete para incluir la bibliografía
\usepackage[backend=bibtex,style=chem-acs,biblabel=dot]{biblatex}
\addbibresource{references.bib}

%Formato del título de las secciones

\usepackage{titlesec}
\usepackage{enumitem}
\titleformat*{\section}{\bfseries\large}
\titleformat*{\subsection}{\bfseries\normalsize}

%Creación del ambiente anexos
\usepackage{float}
\floatstyle{plaintop}
\newfloat{anexo}{thp}{anx}
\floatname{anexo}{Anexo}
\restylefloat{anexo}
\restylefloat{figure}

%Modificación del formato de los captions
\usepackage[margin=10pt,labelfont=bf]{caption}

%Paquete para incluir comentarios
\usepackage{todonotes}

%Paquete para incluir hipervínculos
\usepackage[colorlinks=true,
            linkcolor = blue,
            urlcolor  = blue,
            citecolor = black,
            anchorcolor = blue]{hyperref}

%%%%%%%%%%%%%%%%%%%%%%
%Inicio del documento%
%%%%%%%%%%%%%%%%%%%%%%

\begin{document}
\renewcommand{\labelitemi}{$\checkmark$}

\renewcommand{\CancelColor}{\color{red}}

\newcolumntype{L}[1]{>{\raggedright\let\newline\\\arraybackslash}m{#1}}

\newcolumntype{C}[1]{>{\centering\let\newline\\\arraybackslash}m{#1}}

\newcolumntype{R}[1]{>{\raggedleft\let\newline\\\arraybackslash}m{#1}}

\begin{center}
  %%%%%%%%%%%%%%%%%%%%%%%%%%%%%%%%%%%%%%
  % Ingresa el mejor titulo que se te ocurra
  %%%%%%%%%%%%%%%%%%%%%%%%%%%%%%%%%%%%%%
	\textbf{\LARGE{Listados de preguntas/problemas}}\\
	\vspace{4mm}
		\textbf{\large{Tomás Vera\\
  email: \href{mailto:vtomasv@gmail.com}{vtomasv@gmail.com}  } }\\
	\vspace{3mm}
	\textbf{\large{Universidad de Chile}}\\
	\textbf{\small{CC71T-1 Investigación en Cs. de la Computación.(Métodos,Técnicas,Persp.) }}\\
	\textbf{\large{Profesor: Claudio Gutierrez \\
  email: \href{mailto:cgutierr@dcc.uchile.cl}{cgutierr@dcc.uchile.cl}  } }\\
	\today
\end{center}

%%%%%%%%%%%%%%%%%%%%%%%%%%%%%%%%%%%%%%
% Resumen formal de la noticia
%%%%%%%%%%%%%%%%%%%%%%%%%%%%%%%%%%%%%%
\section*{\centering Resumen}
A continuación se describen tres problemas y/o preguntas que tienen como campo de resolución, las ciencias de la computación. Estas preguntas o problemas son presentadas al profesor Claudio Gutierrez para su estudio y refinamiento con el objetivo de poder seleccionar una de estas y llevar su investigación durante el semestre.

%%%%%%%%%%%%%%%%%%%%%%%%%%%%%%%%%%%%%%
% Tu voz sobre el impacto de esta noticia
%%%%%%%%%%%%%%%%%%%%%%%%%%%%%%%%%%%%%%
\section{Contexto}

“Sólo podemos ver poco del futuro, pero lo suficiente para darnos cuenta de que hay mucho que hacer” - Alan Turing.
Para dar inicio a una investigación lo primero es poder tener por escrito las razones por las que se debe realizar la investigación. A esto podemos llamarle la delimitación del problema, indicando las razones que originan la necesidad de investigar, enunciando el problema, planteando las preguntas que destacan  el planteamiento del problema, justificando la necesidad de hacer la investigación, indicando su viabilidad, delimitando de forma correcta el problema con el fin de acotar la investigación, adicionalmente se debe dar indicios de su solución y de como se realizara la aceptación formal de la solución como correcta.
Con estas premisas se plantearan las siguientes tres preguntas y/o problemas con el fin de seleccionar una para su investigación.


\section{Recuperar autos robados en Chile}

Uno de los delitos más comunes en Latinoamérica son los robos de vehículos\autocite{robos}, en Santiago la tasa de recuperación de vehículos es cercana al 80\% \autocite{recupero}, sin embargo antes de ser recuperados son utilizados para realizar otros tipos de ilícitos lo que perjudica a sus dueños reales. Poder contar con un mecanismo de recuperación temprana permitirá bajar la tasa de delitos efectuados con autos robados. Una de las tecnicas utilizada por las policías de Santiago de Chile es hacer uso de cámaras inteligentes\autocite{recono} que verifican en tiempo real si la placa patente del vehículo que los precede esta reportada como robada o no, contar con este mecanismo de manera universal permitirá contar con mas puntos de observación que faciliten las tareas de identificación y recuperación de vehículos robados.

\section{Optimizar estimaciones de construcción de software bajo la metodología Scrum en al industria de fabricas de software Pyme en Chile}
La estimación de tiempos\autocite{estima}, costo y alcance en el desarrollo de software bajo la metodología Scrum es uno de los factores de desviación de estos proyectos, contar con un framework mas una metodología que incrementalmente y colaborativamente permitan optimizar estas estimaciones bajo la metodología Scrum.


\section{Asegurar la inviolabilidad de resultados de examenes de HIV en Chile }

Poder contar con un mecanismo inviolable que permita asegurar la comunicación  segura de los resultados de exámenes de HIV, seguritizando toda la cadena de comunicación desde la obtención de los resultados hasta la comunicación con el paciente.

%%%%%%%%%%%%%%%%%%%%%%%%%%%%%%%%%%%%%%
% Referencias sobre la noticia
%%%%%%%%%%%%%%%%%%%%%%%%%%%%%%%%%%%%%%
\section{Referencias\label{sec:references}}

\printbibliography[heading=none]

\section{Tu opinion es muy importante!}
\begin{figure}
    \centering
    \includegraphics[width=4cm]{./images/vote.png}
    \captionsetup{justification=centering, singlelinecheck=false}
\end{figure}

\end{document}
