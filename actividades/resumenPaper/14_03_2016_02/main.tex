\documentclass[12pt,letterpaper]{article}
\usepackage[utf8]{inputenc}
\usepackage[spanish, es-tabla]{babel}
\usepackage[version=3]{mhchem}
\usepackage[journal=jacs]{chemstyle}
\usepackage{amsmath}
\usepackage{amsfonts}
\usepackage{amssymb}
\usepackage{makeidx}
\usepackage{xcolor}
\usepackage[stable]{footmisc}
\usepackage[section]{placeins}
%Paquetes necesarios para tablas
\usepackage{longtable}
\usepackage{array}
\usepackage{xtab}
\usepackage{multirow}
\usepackage{colortab}
%Paquete para el manejo de las unidades
\usepackage{siunitx}
\sisetup{mode=text, output-decimal-marker = {,}, per-mode = symbol, qualifier-mode = phrase, qualifier-phrase = { de }, list-units = brackets, range-units = brackets, range-phrase = --}
\DeclareSIUnit[number-unit-product = \;] \atmosphere{atm}
\DeclareSIUnit[number-unit-product = \;] \pound{lb}
\DeclareSIUnit[number-unit-product = \;] \inch{"}
\DeclareSIUnit[number-unit-product = \;] \foot{ft}
\DeclareSIUnit[number-unit-product = \;] \yard{yd}
\DeclareSIUnit[number-unit-product = \;] \mile{mi}
\DeclareSIUnit[number-unit-product = \;] \pint{pt}
\DeclareSIUnit[number-unit-product = \;] \quart{qt}
\DeclareSIUnit[number-unit-product = \;] \flounce{fl-oz}
\DeclareSIUnit[number-unit-product = \;] \ounce{oz}
\DeclareSIUnit[number-unit-product = \;] \degreeFahrenheit{\SIUnitSymbolDegree F}
\DeclareSIUnit[number-unit-product = \;] \degreeRankine{\SIUnitSymbolDegree R}
\DeclareSIUnit[number-unit-product = \;] \usgallon{galón}
\DeclareSIUnit[number-unit-product = \;] \uma{uma}
\DeclareSIUnit[number-unit-product = \;] \ppm{ppm}
\DeclareSIUnit[number-unit-product = \;] \eqg{eq-g}
\DeclareSIUnit[number-unit-product = \;] \normal{\eqg\per\liter\of{solución}}
\DeclareSIUnit[number-unit-product = \;] \molal{\mole\per\kilo\gram\of{solvente}}
\usepackage{cancel}
%Paquetes necesarios para imágenes, pies de página, etc.
\usepackage{graphicx}
\usepackage{lmodern}
\usepackage{fancyhdr}
\usepackage[left=4cm,right=2cm,top=3cm,bottom=3cm]{geometry}

%Instrucción para evitar la indentación
%\setlength\parindent{0pt}
%Paquete para incluir la bibliografía
\usepackage[backend=bibtex,style=chem-acs,biblabel=dot]{biblatex}
\addbibresource{references.bib}

%Formato del título de las secciones

\usepackage{titlesec}
\usepackage{enumitem}
\titleformat*{\section}{\bfseries\large}
\titleformat*{\subsection}{\bfseries\normalsize}

%Creación del ambiente anexos
\usepackage{float}
\floatstyle{plaintop}
\newfloat{anexo}{thp}{anx}
\floatname{anexo}{Anexo}
\restylefloat{anexo}
\restylefloat{figure}

%Modificación del formato de los captions
\usepackage[margin=10pt,labelfont=bf]{caption}

%Paquete para incluir comentarios
\usepackage{todonotes}

%Paquete para incluir hipervínculos
\usepackage[colorlinks=true,
            linkcolor = blue,
            urlcolor  = blue,
            citecolor = black,
            anchorcolor = blue]{hyperref}

%%%%%%%%%%%%%%%%%%%%%%
%Inicio del documento%
%%%%%%%%%%%%%%%%%%%%%%

\begin{document}
\renewcommand{\labelitemi}{$\checkmark$}

\renewcommand{\CancelColor}{\color{red}}

\newcolumntype{L}[1]{>{\raggedright\let\newline\\\arraybackslash}m{#1}}

\newcolumntype{C}[1]{>{\centering\let\newline\\\arraybackslash}m{#1}}

\newcolumntype{R}[1]{>{\raggedleft\let\newline\\\arraybackslash}m{#1}}

\begin{center}
  %%%%%%%%%%%%%%%%%%%%%%%%%%%%%%%%%%%%%%
  % Ingresa el mejor titulo que se te ocurra
  %%%%%%%%%%%%%%%%%%%%%%%%%%%%%%%%%%%%%%
  \textbf{\small{Resumen}}\\
	\textbf{\LARGE{Is Computer Science Science?}}\\
	\vspace{4mm}
		\textbf{\large{Tomás Vera\\
  email: \href{mailto:vtomasv@gmail.com}{vtomasv@gmail.com}  } }\\
	\vspace{3mm}
	\textbf{\large{Universidad de Chile}}\\
	\textbf{\small{CC71T-1 Investigación en Cs. de la Computación.(Métodos,Técnicas,Persp.) }}\\
	\textbf{\large{Profesor: Claudio Gutierrez \\
  email: \href{mailto:cgutierr@dcc.uchile.cl}{cgutierr@dcc.uchile.cl}  } }\\
	\today
\end{center}

%%%%%%%%%%%%%%%%%%%%%%%%%%%%%%%%%%%%%%
% Resumen formal de la noticia
%%%%%%%%%%%%%%%%%%%%%%%%%%%%%%%%%%%%%%
\section*{\centering Resumen}
¿Hay Ciencia en Ciencia de la Computación o solo es una ilusión? Desde el 1950 que se viene acuñando el termino de Ciencia de la Computación, con toda la controversia que esto representa, desde esa época, ciencia, ingeniería y matemática se combinan en una única y potente mezcla llamada Ciencia de la Computación. En este articulo Peter J. Denning\autocite{Peter2005} intenta poner de manifiesto que es la ciencia?, como se relaciona esta con la Computación?, como es la Ciencia de la Computación en acción?, intentando sobreponerse no solo a los cuestionamientos internos sino también a los distintos detractores de desalojar a las Ciencias de la Computación de su estrado y relegan solo a una disciplina o simplemente una rama de alguna de las ciencias\autocite{ciencia}. Peter nos introduce en un dialogo en el cual intenta a través de la explicación de las actividades de un cientista\autocite{Cientista} de la Computación como estas actividades son netamente ciencia por ejemplo, algoritmos experimentales,
ciencia de la computación experimental y ciencia computacional, como otras actividades son netamente ingeniería, por ejemplo, diseño, desarrollo, ingeniería de software e ingeniería de computación o simplemente son primariamente matemáticas, por ejemplo, complejidad de algoritmos, software
matemático y análisis numérico. Todo esto en el marco de comprender las ciencias como tal y su aplicacion con las Ciencias de la Computación. Una vez definida la ciencia como tal realiza un certera comparación con el arte, buscando desambiguar estas dos, en su comparación deja clara su vision sobre diferencias estructurales de la ciencia y el arte reforzando este concepto en la definición de que el arte tiene practica versus la ciencia que tiene principios, pasando a detallar en su vision los principales principios de esta Ciencia. Para finalizar profundiza como la Ciencia de la computación aun puede sorprender, no solo por su naturaleza como ciencia, sino que también por el camino de su inter relación con las demás ciencias. Como ultimo punto pero no menos fundamental Peter se hace cargo del peso histórico de la poca rigurosidad en los textos de la Ciencia de la Computación, explicando las medidas y nuevas formas de hacer ciencia apoyado en Tichy\autocite{Peter2005} que indica que la reciente literatura científica muestra un marcado aumento en testes.

%%%%%%%%%%%%%%%%%%%%%%%%%%%%%%%%%%%%%%
% Tu voz sobre el impacto de esta noticia
%%%%%%%%%%%%%%%%%%%%%%%%%%%%%%%%%%%%%%
\section{Opinion}
Personalmente me enfrente al mundo de la computacion muy joven, en 1991 y a mi corta edad (10 años) no tenia noción de que era o no ciencia, lo que sentía era una curiosidad inmensurable por poder hacer cosas que soñaba que eran posibles con una computadora, como por ejemplo, que estas sean inteligentes (por 2001:Odisea del Espacio\autocite{SpaceOdyssey}), que puedan hablar con las personas, que manejen nuestro hogar, etc. Más grande (no indica maduro) en la Universidad empece a adquirir los conceptos de la ciencia y a mezclarlos con el arte, dado que por alguna razón sentía que todo aquello que inspira sentimientos como un cuadro, una canción es arte, la ciencia de la computacion para mi, inspira eso, sentimientos. Despues de leer este articulo pensé en MacGyver\autocite{MacGyver}, si en el, para los mas jóvenes, MacGyver es un agente al servicio de la Fundación Phoenix que siempre resuelve todos los problemas usando su inteligencia superior y sus amplios conocimientos técnicos. Que seria de el sin su navaja suiza\autocite{NavajaSuiza}, como lo demostró en uno de sus capítulos\autocite{usarClip}, solo basta tener un clip\autocite{clip}\autocite{kitDeSupervivencia} ya que la navaja no era el fin sino un instrumento, creo que la comunidad tiende a confundirnos con MacGyver, la computadora o computador no es nuestra navaja suiza, nuestro potencial esta en nuestra ciencia, en los principios que esta tiene, en su potencial de crecimiento y en la capacidad de sorprender día a día, claramente como MacGyver en su momentos somos unos agentes jóvenes, 70 años no es nada y por eso estamos formalizando cada día mas nuestros avances, siendo cada vez mas rigurosos en esto, que no solo llamamos sino que sentimos en todo nuestro cuerpo, la Ciencia de la Computación.

%%%%%%%%%%%%%%%%%%%%%%%%%%%%%%%%%%%%%%
% Referencias sobre la noticia
%%%%%%%%%%%%%%%%%%%%%%%%%%%%%%%%%%%%%%
\section{Referencias\label{sec:references}}

\printbibliography[heading=none]

\section{Tu opinion es muy importante!}
\begin{figure}
    \centering
    \includegraphics[width=4cm]{./images/vote.png}
    \captionsetup{justification=centering, singlelinecheck=false}
\end{figure}


\end{document}
